\phantomsection
\setsection{Chương 1: Tổng quan}
\setcounter{section}{1}

% \section*{\centering CHƯƠNG 1: TỔNG QUAN}
\phantomsection
\subsection{Mô tả về bài toán}
Bài toán người giao hàng (Travelling Salesman Problem - TSP) là một trong những vấn đề tối ưu hóa kinh điển trong khoa học máy tính và toán học. Mục tiêu của bài toán này là tìm ra hành trình ngắn nhất mà một người giao hàng cần đi qua để ghé thăm tất cả các địa điểm (các điểm giao hàng) và trở về điểm xuất phát ban đầu, với điều kiện mỗi địa điểm chỉ được ghé thăm đúng một lần.

Bài toán cây khung nhỏ nhất (Minimum Spanning Tree - MST) là một bài toán cơ bản trong lý thuyết đồ thị, có ứng dụng rộng rãi trong các lĩnh vực như mạng máy tính, các hệ thống kết nối, viễn thông, và thiết kế mạch điện. Bài toán yêu cầu tìm cây khung nhỏ nhất của một đồ thị, tức là tìm ra một tập hợp các cạnh kết nối tất cả các đỉnh với nhau mà không hình thành chu trình, đồng thời có tổng trọng số của các cạnh là nhỏ nhất.

\phantomsection
\subsection{Mục tiêu của đề tải}
Đề tài này nhằm mục tiêu xây dựng một công cụ để giải quyết các bài toán tối ưu hóa như bài toán người giao hàng và bài toán cây khung nhỏ nhất một cách nhanh chóng và chính xác. Đồng thời, đề tài sẽ xây dựng các giải thuật tối ưu dựa trên kiến thức đã học, mô tả chi tiết phương pháp giải quyết các bài toán theo thuật toán đã triển khai, và cung cấp giao diện đồ thị trực quan cho người dùng.

\phantomsection
\subsection{Hướng giải quyết và kế hoạch thực hiện}